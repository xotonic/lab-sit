
% Article - значит без глав, только секции
\documentclass{article}
\usepackage{lmodern}
% Задаем шрифты
\usepackage{fontspec}
\setmainfont[Scale=1.4]{FreeSerif} % Главный шрифт
\setmonofont[Scale=1.0]{Ubuntu Mono} % Моноширинный шрифт

% Поддержка русского языка (переносов, орфографии)
\usepackage{polyglossia}
\setmainlanguage{russian} % Ставим русский главным
\setotherlanguage{english} % Английский второстепенным

% Отступы по ГОСТу
\usepackage{geometry}
\geometry{
	a4paper,
	left=3cm,
	top=2cm,
	bottom=2cm,
	right=1cm,
}

% Можно вставлять скрипты на Lua (компилить через lualatex)
%\usepackage{luacode}

% Example
% \begin{luacode}
% tex.print(math.random())
% \end{luacode}

\usepackage{listings}
\lstset{
  basicstyle=\ttfamily,
  keywordstyle=\ttfamily,
  stringstyle=\ttfamily,
  commentstyle=\ttfamily,
  breaklines=true,
  keepspaces=true,
  extendedchars=\true
}


\usepackage{python}


%%% ТИТУЛЬНИК %%%
%%%%%%%%%%%%%%%%%

\begin{document}
% Отключаем нумерацию страниц
\pagenumbering{gobble}
\begin{center}
МИНИСТЕРСТВО ОБРАЗОВАНИЯ И НАУКИ РОССИЙСКОЙ ФЕДЕРАЦИИ
\vspace{20pt}

ФЕДЕРАЛЬНОЕ  АГЕНТСТВО  ПО  ОБРАЗОВАНИЮ

ФЕДЕРАЛЬНОЕ  ГОСУДАРСТВЕННОЕ  БЮДЖЕТНОЕ ОБРАЗОВАТЕЛЬНОЕ  УЧРЕЖДЕНИЕ 
ВЫСШЕГО  ПРОФЕССИОНАЛЬНОГО  ОБРАЗОВАНИЯ

НОВОСИБИРСКИЙ  ГОСУДАРСТВЕННЫЙ  ТЕХНИЧЕСКИЙ  УНИВЕРСИТЕТ

\vspace{\fill}
{\bfseries \Large Лабораторная работа № 3}

{\itshape по дисциплине <<Современные информационные технологии>>}

на тему "Многопотоковые приложения. Потоки ввода-вывода.
         Сериализация объектов в файл."

\vspace{\fill}

\begin{flushleft}
\begin{tabular}{ l l }
Студент & Кузьмин Д.С. \\
Группа & АВТ-318 \\
Преподаватель & Васюткина И.А. \\
Вариант & 8 \\
\end{tabular}
\end{flushleft}

\vspace{\fill}
Новосибирск 2015 г.
\end{center}
\pagebreak

% Включаем нумерацию страниц
\pagenumbering{arabic}

%%% СЕКЦИИ %%%
%%%%%%%%%%%%%%

\section*{Цель работы}

\begin{enumerate}
\item Изучить особенности реализации и работы потоков в Java, управлением приоритетами потоков и синхронизацией потоков.
\item Доработать программу, созданную в лабораторных работах № 1-2:
\end{enumerate}


\section*{Задание варианта}

Вариант задания:
\begin{enumerate}
\item Автомобили двигаются по оси X от одного края области симуляции до другого со скоростью V.
\item Мотоциклы двигаются по оси Y от одного края области симуляции до другого со скоростью V.
\end{enumerate}
\section*{Задание}

Доработать программу, созданную в лабораторной работе № 2:
\begin{enumerate}

\item Создать абстрактный класс AI, описывающий «интеллектуальное поведение» объектов по варианту. Класс должен быть выполнен в виде отдельного потока и работать с коллекцией объектов;
\item Реализовать класс BaseAI для каждого из видов объекта, включив в него поведение, описанное в индивидуальном задании по варианту;
\item Синхронизовать работу потоков расчета интеллекта объектов с их рисованием. Рисование должно остаться в основном потоке. Синхронизация осуществляется через передачу данных в основной поток;
\item Добавить в панель управления кнопки для остановки и возобновления работы интеллекта каждого вида объектов. Реализовать через засыпание/пробуждение потоков;
\item Добавить в панель управления выпадающие списки для выставления приоритетов каждого из потоков.
\item Реализовать сохранение объектов в файл.
\item Реализовать сохранение и загрузку настроек параметров программы в локальный файл.

\end{enumerate}

\section*{Вывод}
Были рассмотрены особенности реализации и работы потоков в Java, управлением приоритетами потоков и синхронизацией потоков.
Была разработана многооточная программа для симуляции движение транспортных средств, в которой логика, ИИ и визуализация
просчитываются в различных потоках и реализовано их синхронизирванное взаимодействие между собой.

\section*{Приложение А. Листинг программы}

%code%

\end{document}
