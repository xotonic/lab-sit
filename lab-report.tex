
% Article - значит без глав, только секции
\documentclass{article}
\usepackage{lmodern}
% Задаем шрифты
\usepackage{fontspec}
\setmainfont[Scale=1.4]{FreeSerif} % Главный шрифт
\setmonofont[Scale=1.0]{Ubuntu Mono} % Моноширинный шрифт

% Поддержка русского языка (переносов, орфографии)
\usepackage{polyglossia}
\setmainlanguage{russian} % Ставим русский главным
\setotherlanguage{english} % Английский второстепенным

% Отступы по ГОСТу
\usepackage{geometry}
\geometry{
	a4paper,
	left=3cm,
	top=2cm,
	bottom=2cm,
	right=1cm,
}

% Можно вставлять скрипты на Lua (компилить через lualatex)
%\usepackage{luacode}

% Example
% \begin{luacode}
% tex.print(math.random())
% \end{luacode}

\usepackage{listings}
\lstset{
  basicstyle=\ttfamily,
  keywordstyle=\ttfamily,
  stringstyle=\ttfamily,
  commentstyle=\ttfamily,
  breaklines=true,
}


\usepackage{python}


%%% ТИТУЛЬНИК %%%
%%%%%%%%%%%%%%%%%

\begin{document}
% Отключаем нумерацию страниц
\pagenumbering{gobble}
\begin{center}
МИНИСТЕРСТВО ОБРАЗОВАНИЯ И НАУКИ РОССИЙСКОЙ ФЕДЕРАЦИИ
\vspace{20pt}

ФЕДЕРАЛЬНОЕ  АГЕНТСТВО  ПО  ОБРАЗОВАНИЮ

ФЕДЕРАЛЬНОЕ  ГОСУДАРСТВЕННОЕ  БЮДЖЕТНОЕ ОБРАЗОВАТЕЛЬНОЕ  УЧРЕЖДЕНИЕ 
ВЫСШЕГО  ПРОФЕССИОНАЛЬНОГО  ОБРАЗОВАНИЯ

НОВОСИБИРСКИЙ  ГОСУДАРСТВЕННЫЙ  ТЕХНИЧЕСКИЙ  УНИВЕРСИТЕТ

\vspace{\fill}
{\bfseries \Large Лабораторная работа № 2}

{\itshape по дисциплине <<Современные информационные технологии>>}

на тему "Разработка графического интерфейса. 
Классы-коллекции.
Паттерны проектирования поведения объектов"
\vspace{\fill}

\begin{flushleft}
\begin{tabular}{ l l }
Студент & Кузьмин Д.С. \\
Группа & АВТ-318 \\
Преподаватель & Васюткина И.А. \\
Вариант & 8 \\
\end{tabular}
\end{flushleft}

\vspace{\fill}
Новосибирск 2015 г.
\end{center}
\pagebreak

% Включаем нумерацию страниц
\pagenumbering{arabic}

%%% СЕКЦИИ %%%
%%%%%%%%%%%%%%

\section*{Цель работы}

\begin{enumerate}
\item Познакомиться с основными компонентами построения графических интерфейсов библиотек AWT и Swing в программах на Java. Изучить классы менеджеров компоновки.
\item Изучить назначение классов-коллекций, их виды, и методы работы с классами-коллекциями.
\end{enumerate}


\section*{Задание варианта}

Вариант задания:

Список транспортных средств на дороге состоит из двух категорий: автомобили и мотоциклы. Автомобили генерируются каждые $N_{1}$ секунд с вероятностью $P_{1}$. Мотоциклы генерируются каждые $N_{2}$ секунд с вероятностью $P_{2}$.

\section*{Задание}

Доработать программу, созданную в лабораторной работе № 1:
\begin{enumerate}

\item Поделить рабочую область окна приложения на 2 части. Визуализация переносится в одну часть окна, панель управления в другую;

\item Добавить кнопки «Старт» и «Стоп» в панель управления. Они должны запускать и останавливать симуляцию соответственно. Если симуляция остановлена, то кнопка «Стоп» должна блокироваться. Если симуляция идет, то блокируется кнопка «Старт». Клавиши B и E должны функционировать по-прежнему;

\item Добавить переключатель «Показывать информацию», который разрешает отображение модального диалога из 7 пункта задания;

\item Добавить группу из 2 исключающих переключателей: «Показывать время симуляции» и «Скрывать время симуляции». Клавиша T должна функционировать по-прежнему;

\item Используя различные менеджеры компоновки, сформировать интерфейс пользователя согласно индивидуальному заданию;

\item Добавить в программу главное в меню и панель инструментов, в которых продублировать основные команды вашего интерфейса пользователя;

\item При остановке симуляции должно появляться модальное диалоговое окно (при условии, что оно разрешено) с информацией о количестве и типе сгенерированных объектов, а также времени симуляции. Вся информация выводится в элементе TextArea, недоступном для редактирования. В диалоговом окне должно быть 2 кнопки: <<ОК>> и <<Отмена>>. При нажатии на <<ОК>> симуляции останавливается, а при нажатии на <<Отмена>>, соответственно продолжается;

\item Предусмотреть проверку данных вводимых пользователем. При вводе неверного значения обрабатывать исключительную ситуацию: выставлять значение по умолчанию и выводить диалоговое окно с сообщением об ошибке;
\item Реализовать следующие элементы управления:
\begin{itemize}
	\item Периоды рождения объектов – текстовые поля;
	\item Для задания вероятностей рождения  объектов комбобокс и  список (шаг значений 	10\%);
	\item Дополнить интерфейс поясняющими метками.
\end{itemize}
\end{enumerate}

\section*{Приложение А. Листинг программы}

%code%

\section*{Вывод}
Произошло ознакомление с особенностями технологии Java и была изучена часть синтаксиса языка Java. Была разработана программа для упрощенной имитации поведения объектов.

\end{document}
